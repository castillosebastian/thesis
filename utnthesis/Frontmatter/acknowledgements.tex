
En el campo del aprendizaje automático, la selección de características es una tarea crítica que puede determinar el éxito o fracaso de un modelo predictivo. La alta dimensionalidad y la complejidad inherente a los conjuntos de datos reales hacen que la selección de un subconjunto óptimo de características sea, muchas veces, un paso ineludible para el aprendizaje efectivo.

En este contexto, los Algoritmos Genéticos (AG) se han consolidado como una herramienta poderosa para resolver problemas de optimización complejos, incluida la selección de características (Vignolo y Gerard 2017). Estos algoritmos, inspirados en la evolución natural, son capaces de explorar grandes espacios de búsqueda de manera efectiva, proporcionando soluciones suficientemente buenas en una amplia variedad de escenarios. Por ello, los AG han sido ampliamente utilizados en problemas de selección, demostrando su eficacia en la identificación de subconjuntos relevantes de características en datos de alta dimensionalidad.

Sin embargo, la eficacia de los AG depende de la disponibilidad de suficientes datos para evaluar las soluciones en competencia. En contextos donde los datos son limitados, donde cualquier método de selección de características es sensible a esta limitación, también los AG pueden verse afectados en su capacidad para seleccionar características relevantes, produciendo soluciones subóptimas o inestables. Esta limitación es especialmente crítica en problemas de alta dimensionalidad y bajo número de muestras, donde la función objetivo que guía la búsqueda de soluciones puede degradarse significativamente (Bolón-Canedo, Sánchez-Maroño, y Alonso-Betanzos 2015).

Por esta razón, la investigación de estrategias que mitiguen las limitaciones impuestas por la escasez de datos se ha convertido en un área de interés creciente en el subcampo de la selección de características. Una de las técnicas emergentes en este ámbito es la aumentación de datos mediante Autocodificadores Variacionales (AV). Los AV, como modelos generativos, tienen la capacidad de crear muestras sintéticas que mantienen las propiedades fundamentales de los datos originales, conviertiéndolos en una herramienta prometedora para mejorar la capacidad de los AG en la selección de características.

El problema central de la tesis que aquí presentamos gira, precísamente, en la restricciones que la escacez de datos impone a la función objetivo que guía la búsqueda de soluciones de los AG, y cómo superarlas usando AV. Así, la pregunta central del trabajo es: ¿cómo puede la aumentación de datos mediante autocodificadores variacionales mejorar el desempeño de los algoritmos genéticos en la selección de características?

Cabe destacar que este desafío y su eventual solución son importantes por varias razones. La selección de características no solo condiciona, como ya se mencionó, la precisión de los modelos predictivos, también afecta la eficiencia computacional y la interpretabilidad de los resultados. En problemas de alta dimensionalidad, la posibilidad de reducir el número de características relevantes sin perder información útil puede marcar la diferencia entre un modelo efectivo y uno ineficaz, entre uno interpretable y uno de caja negra. Por lo tanto, mejorar este proceso mediante la integración de técnicas de aumentación de datos puede tener un impacto significativo en diversas aplicaciones prácticas, desde la biología molecular hasta la ingeniería y las ciencias sociales (El-Hasnony et al. 2020).

La hipótesis que hemos llevado a prueba ha sido que la aumentación de datos mediante AV mejora la capacidad de selección de características de los AG, favoreciendo la evaluación de soluciones en competencia y permitiendo la identificación de subconjuntos de características más relevantes y estables en contextos de escasez muestral. Para evaluarla, hemos propuesto un trabajo experimental que exploró la integración de estas dos técnicas en un marco unificado. La idea de combinar la generación de datos sintéticos mediante AV con la selección de características mediante AG, estaba orientada a buscar combinaciones sinérgicas y eficaces entre modelos que mejorasen la selección de características. A estos fines, trabajamos con cinco conjuntos de datos de referencia, representativos de distintos contextos y niveles de complejidad, para evaluar el desempeño de los modelos propuestos.

A lo largo de este documento, describiremos el proceso de investigación llevado a cabo, desde los estudios iniciales hasta los experimentos finales, pasando por el diseño y construcción de un modelo genérico de AV, y su adaptación a los datasets elegidos y la creación de una estructura combinada de AV + AG para la selección de características. Los resultados obtenidos en cada etapa se analizarán y discutirán en detalle, con el objetivo de identificar las ventajas y limitaciones de la propuesta, así como posibles áreas de mejora y futuros trabajos.

Al finalizar el documento, esperamos poder justificar la eficacia de la aumentación de datos mediante AV en la selección de características, demostrando que esta técnica puede mejorar significativamente el desempeño en contextos de escasez. Además, esperamos identificar las condiciones y contextos en los que esta técnica es más efectiva.

El documento está estructurado de la siguiente manera: en el Capítulo 1 se presenta el problema de la selección de características en contextos de alta dimensionalidad y escacez muestral. En el Capítulo 2 se hace una revisión de modelos clásicos aplicados a los datos que forman parte de nuestra investigación, su evaluación y resultados. En el Capítulo 3 se presentan los modelos AV, sus bases teóricas y los experimentos realizados para construir la arquitectura final empleada en nuestra investigación. En el Capítulo 4 se aborda la integración de los modelos AV y AG en una estructura combinada, su configuración y los resultados experimentales obtenidos de su aplicación. Finalmente, en el Capítulo 5, se presentan las conclusiones de la investigación, así como posibles líneas futuras de trabajo.

Por último, en tiempos de grandes debates sobre los riesgos de los modelos generativos, esperamos poder brindar a nuestros lectores un ejemplo funcional de su estructura, iluminar ciertos principios y evidenciar sus límites. Si nuestro trabajo permite resaltar la sobriedad de sus mecanismos internos, habremos cumplido el anhelo de hacer menos opaca su belleza y más clara sus posibilidades.
